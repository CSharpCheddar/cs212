
\documentclass[12pt]{exam}

\usepackage[margin=1in]{geometry}
\usepackage{amsmath,amssymb}
\usepackage{color}
\usepackage{fancyvrb}

\newcommand{\class}{CS 212}
\newcommand{\term}{Spring 2019}
\newcommand{\examnum}{Assignment 8}
\newcommand{\examdate}{4/30/2019}

\pagestyle{head}
\firstpageheader{}{}{}
\runningheader{\class}{\examnum\ - Page \thepage\ of \numpages}{\examdate}
\runningheadrule


\begin{document}

\noindent
\begin{tabular*}{\textwidth}{l @{\extracolsep{\fill}} r @{\extracolsep{6pt}} l}
\textbf{\class} & \textbf{DUE: Tuesday, May 7 at 11:59 PM} & \\
\textbf{\term} &&\\
\textbf{\examnum} &&\\
\textbf{\examdate} &\textbf{10 points}&\\
\end{tabular*}\\
\rule[2ex]{\textwidth}{2pt}

\paragraph{Introduction:} This assignment will give you practice working with summations and recurrence relations. \\
\\
This is an individual effort homework assignment. You must write up your solutions in \LaTeX. Use the {\tt a8.tex} template that I provide and be sure to replace each ``Put your answer for \makebox[.25in]{\hrulefill} here.'' with your answers but leave everything else alone. Your solutions must be written up in a clear, concise and rigorous manner.  \\
\\
\noindent When you are done, zip up your .TEX file and corresponding .PDF file. Upload your .ZIP file to the {\bf a8} dropbox on d2l. After you have uploaded the file, double-check to ensure your file was uploaded correctly. It is your responsibility to ensure your submission was done correctly.  Assignments that are not uploaded correctly are worth 0 points. \\
\\
\\
\\
\\
\\
\\
Consider the following recursive Java method: 
\begin{verbatim}
   // Assumes n is a non-negative integer, e.g., 0, 1, ... 
   public static int myMethod(int n) {
      if (n == 0) return 1;
      int sum = 0;
      for (int i = 0; i <= n - 1; i++) {
         sum += myMethod(i) + myMethod(i);
      }
      return 1 + sum;
   }
\end{verbatim} 

\noindent We would like to study {\tt myMethod} and determine a closed-form expression for its return value, for all non-negative {\tt int} values. We will do this by first formulating a recurrence relation for the value returned by {\tt myMethod}, and then we will use iteration to solve for a closed-form expression. Going forward, let $T(n)$ denote the value returned by {\tt myMethod}. 








\clearpage

\begin{questions}


	\question[1] \label{qBase} What is the base case for for $T(n)$?
	\begin{choices}
		\choice $T(0) = 0$ 				%A
		\choice $T(0) = 1$ 				%B
		\choice $T(1) = 0$ 				%C
		\choice $T(1) = 1$ 				%D
	\end{choices}

	\fbox{\begin{minipage}[t]{14.55cm} \color{red}
			\vspace{25pt}
				
			\begin{center}
				B. $T(0) = 1$
			\end{center}
				
			\vspace{25pt}
		\end{minipage}
	}
		
\clearpage

	\question[1] \label{qRec} What is the recursive case for $T(n)$?
	\begin{choices}
		\choice $\displaystyle T(n) = 1 + 2\sum_{i=0}^{n-1}{T(i)}$					%A
		\choice $\displaystyle T(n) = 1 + \sum_{i=1}^{n-1}{\left(T(i)+T(i)\right)}$	%B
		\choice $\displaystyle T(n) = \sum_{i=0}^{n-1}{2\cdot T(i)}$				%C
		\choice $\displaystyle T(n) = 1 + 2\sum_{i=0}^{n}{T(i)}$					%D
	\end{choices}

	\fbox{\begin{minipage}[t]{14.55cm} \color{red}
			\vspace{25pt}
				
			\begin{center}
				A. $\displaystyle T(n) = 1 + 2\sum_{i=0}^{n-1}{T(i)}$
			\end{center}
				
			\vspace{25pt}
		\end{minipage}
	}
	
	
	\vspace{100pt}
	Work the rest of the problems assuming your choices for questions~\ref{qBase} and~\ref{qRec}. 

\clearpage

	\question[1] Compute $T(1)$:
	
		
	\fbox{\begin{minipage}[t]{14.55cm} \color{red}
			\vspace{25pt}
				
			\begin{align*}
				T(1) &= 1 + 2 \sum_{i=0}^{1-1} T(i) \\
				&= 1 + 2(1) \\
				&= \boxed{3}
			\end{align*}
				
			\vspace{25pt}
		\end{minipage}
	}
	
\clearpage


	\question[1] Compute $T(2)$:
	
		
	\fbox{\begin{minipage}[t]{14.55cm} \color{red}
			\vspace{25pt}
				
			\begin{align*}
				T(2) &= 1 + 2 \sum_{i=0}^{2-1} T(i) \\
				&= 1 + 2(1 + 3) \\
				&= \boxed{9}
			\end{align*}
				
			\vspace{25pt}
		\end{minipage}
	}
	
\clearpage

	
	\question[1] Compute $T(3)$:
	
		
	\fbox{\begin{minipage}[t]{14.55cm} \color{red}
			\vspace{25pt}
				
			\begin{align*}
				T(3) &= 1 + 2 \sum_{i=0}^{3-1} T(i) \\
				&= 1 + 2(1 + 3 + 9) \\
				&= \boxed{27}
			\end{align*}
				
			\vspace{25pt}
		\end{minipage}
	}
	

	
\clearpage
	
	
	\question[1] \label{qClosedForm} Using $T(0)$, $T(1)$, $T(2)$, and $T(3)$, guess a closed-form expression for $T(n)$.
	
		
	\fbox{\begin{minipage}[t]{14.55cm} \color{red}
			\vspace{25pt}
			
			\begin{align*}
				T(0) &= 1 \\
				T(1) &= 3 \\
				T(2) &= 9 \\
				T(3) &= 27
			\end{align*}
			$$\boxed{V_{n} = 3^{n}}$$
				
			\vspace{25pt}
		\end{minipage}
	}
	
\clearpage



	\question[4] Verify your guessed solution from question~\ref{qClosedForm} is correct.
	
		
	\fbox{\begin{minipage}[t]{14.55cm} \color{red}
			\vspace{25pt}
				
			\begin{align*}
				V_{0} &= 3^{0} \\
				&= 1 \\
				&\stackrel{\checkmark}{=} T(0)
			\end{align*}
			\begin{center}
				Now we'll prove the recursive case. First, let's start by looking at the definition of $T(n)$:
				$$T(n) = 1 + 2\sum_{i=0}^{n-1}T(i)$$
				Just like with weak induction, let's assume that $T(m) = V(m), \forall m \le n-1$. That means that we can swap out $T(i)$ for $V(i)$ since the upper limit of the summation is $n-1$:
				$$T(n) = 1 + 2\sum_{i=0}^{n-1}V(i)$$
				Since $V(n) = 3^{n}$, (or in this case, $V(i) = 3^{i}$), we can swap that out in the summation:
				$$T(n) = 1 + 2\sum_{i=0}^{n-1}3^{i}$$
				Now let's rewrite the expression by getting rid of the summation notation:
				$$T(n) = 1 + 2(3^{0} + 3^{1} + 3^{2} + \cdots + 3^{n-1})$$
				As an aside, let's set $2(3^{0} + 3^{1} + 3^{2} + \cdots + 3^{n-1})$ equal to $S$. Now let's see what happens if we multiply $S$ by 3:
				$$3 \cdot 2(3^{0} + 3^{1} + 3^{2} + \cdots + 3^{n-1}) = 2(3^{1} + 3^{2} + 3^{3} + \cdots + 3^{n})$$
				Now we can subtract $S$ from $3S$:
				\begin{align*}
					3S - S &= 2(3^{1} + 3^{2} + 3^{3} + \cdots + 3^{n}) - 2(3^{0} + 3^{1} + 3^{2} + \cdots + 3^{n-1}) \\
					2S &= 2(3^{n} - 3^{0}) \\
					S &= 3^{n} - 1
				\end{align*}
			\end{center}
			
			\vspace{25pt}
		\end{minipage}
	}

\clearpage

\fbox{
	\begin{minipage}[t]{14.55cm} \color{red}
		\vspace{25pt}
			
			\begin{center}
				Here we have a closed form expression for $S$. We can take this and plug it back into the formula for $T(n)$:
				\begin{align*}
					T(n) &= 1 + 3^{n} - 1 \\
					&= 3^{n} \\
					&\stackrel{\checkmark}{=} V(n)
				\end{align*}
			\end{center}
			
		\vspace{25pt}
	\end{minipage}
}
	
	
	
	
	
	
	
	
	
	
	
	
	
		
	
\end{questions}
		


\end{document}
