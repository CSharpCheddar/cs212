
\documentclass[12pt]{exam}

\usepackage[margin=1in]{geometry}
\usepackage{amsmath,amssymb}
\usepackage{color}

\newcommand{\class}{CS 212}
\newcommand{\term}{Spring 2019}
\newcommand{\examnum}{Assignment 6}
\newcommand{\examdate}{4/11/2019}

\pagestyle{head}
\firstpageheader{}{}{}
\runningheader{\class}{\examnum\ - Page \thepage\ of \numpages}{\examdate}
\runningheadrule


\begin{document}

\noindent
\begin{tabular*}{\textwidth}{l @{\extracolsep{\fill}} r @{\extracolsep{6pt}} l}
\textbf{\class} & \textbf{DUE: Thursday, April 18 at 11:59 PM} & \\
\textbf{\term} &&\\
\textbf{\examnum} &&\\
\textbf{\examdate} &\textbf{10 points}&\\
\end{tabular*}\\
\rule[2ex]{\textwidth}{2pt}

\paragraph{Introduction:} This assignment will give you practice working with functions. \\
\\
This is an individual effort homework assignment. You must write up your solutions in \LaTeX. Use the {\tt a6.tex} template that I provide and be sure to replace each ``Put your answer for \makebox[.25in]{\hrulefill} here.'' with your answers but leave everything else alone. Your solutions must be written up in a clear, concise and rigorous manner.  \\
\\
\noindent When you are done, zip up your .TEX file and corresponding .PDF file. Upload your .ZIP file to the {\bf a6} dropbox on d2l. After you have uploaded the file, double-check to ensure your file was uploaded correctly. It is your responsibility to ensure your submission was done correctly.  Assignments that are not uploaded correctly are worth 0 points. \\
\\
\\
\\
\\
\\
\\
\\
\\
\\
\\
\\
\\
\\
Going forward, assume that $$\mathbb{N} = \{0, 1, 2, \ldots\}$$ and $$\mathbb{Q}^{\geq 0}= \left\{ \left. \frac{p}{q} \;\right|\; p\in \mathbb{N} \wedge q \in \mathbb{N} \wedge q\ne 0 \wedge \textmd{gcd}(p,q)=1 \right\}.$$
\\
In this homework, you will study functions with domain and co-domain $\mathbb{N} \times \mathbb{N} \times \mathbb{Q}^{\geq 0}$ and $\mathbb{N}$, respectively.





\clearpage

\begin{questions}

	\question[7] Let $f: \mathbb{N} \times \mathbb{N} \times \mathbb{Q}^{\geq 0} \longrightarrow \mathbb{N}$. 
	\begin{parts}
		\part[1] \label{first} Give an example of two distinct elements of the domain of $f$. 
		
		\fbox{\begin{minipage}[t]{14.55cm} \color{red}
				\vspace{25pt}
				
				\begin{center}
					$(0, 1, \frac{1}{1})$ and $(1, 1, \frac{1}{1})$
				\end{center}
				
				\vspace{25pt}
			\end{minipage}
		}
		
		
\clearpage
		
		
		
		\part[1] Give an example of two distinct elements of the co-domain of $f$.
	
		
		\fbox{\begin{minipage}[t]{14.55cm} \color{red}
			\vspace{25pt}
			
			\begin{center}
				0 and 1
			\end{center}
			
			\vspace{25pt}
			\end{minipage}
		}
	
	
		
\clearpage



		\part[1] Explain why the rule $f\left(x,y,\frac{p}{q}\right) = x+y+p+q$ is not a one-to-one function.
		
		\fbox{\begin{minipage}[t]{14.55cm} \color{red}
			\vspace{25pt}
			
			\begin{center}
				Let $x = 1$, $y = 2$, $p = 1$, and $q = 1$. We can now evaluate $f\left(x,y,\frac{p}{q}\right)$: $f\left(1,2,\frac{1}{1}\right) = 1 + 2 + 1 + 1 = 5$. Now let $x = 2$, $y = 1$, $p = 1$, and $q = 1$. With these different inputs, let's now evaluate $f\left(x,y,\frac{p}{q}\right)$: $f\left(2,1,\frac{1}{1}\right) = 2 + 1 + 1 + 1 = 5$. Since we have arrived at the same output despite having 2 different inputs, this function is not one-to-one. $\square$
			\end{center}
			
			\vspace{25pt}
			\end{minipage}
		}



\clearpage

		\part[1] \label{rule-f} Specify a rule for $f$ that is one-to-one. \\
		
			
		\fbox{\begin{minipage}[t]{14.55cm} \color{red}
			\vspace{25pt}
			
			\begin{center}
				$f\left(x,y,\frac{p}{q}\right) = a \cdot b \cdot c \cdot d$ where $a$, $b$, $c$, and $d$ are all unique prime numbers
			\end{center}
			
			\vspace{25pt}
			\end{minipage}
		}
	
	
	
\clearpage

		\part[1] Take the elements (of the domain of $f$) that you specified in (\ref{first}) and give the elements (of the co-domain of $f$) to which they map under your rule for $f$ from (\ref{rule-f}). \\
		
		
		\fbox{\begin{minipage}[t]{14.55cm} \color{red}
				\vspace{25pt}
				
				\begin{center}
					Since all prime numbers are natural numbers, a product of them will also be a natural number, therefore the codomain of $f$ is $\mathbb{N}$.
				\end{center}
				
				\vspace{25pt}
			\end{minipage}
		}

\clearpage
	
		\part[1] Prove that the rule that you specified in (\ref{rule-f}) is one-to-one. If your rule is not one-to-one, then you will automatically receive 0 for this part. \\
		
		
		\fbox{\begin{minipage}[t]{14.55cm} \color{red}
				\vspace{25pt}
				
				\begin{center}
					By the fundamental theorem of arithmetic, each value of $f$ in the domain would map to a unique value in the codomain of $f$, as $a$, $b$, $c$, and $d$ are all unique prime numbers.
				\end{center}
				
				\vspace{25pt}
			\end{minipage}
		}

	
	\end{parts}
	
	
		
\clearpage



	\question[3] Let $g: \mathbb{N} \times \mathbb{N} \times \mathbb{Q}^{\geq 0} \longrightarrow \mathbb{N}$.

	\begin{parts}
		
		\part[1] \label{rule-g} Specify a rule for $g$ that is onto.
		
		\fbox{\begin{minipage}[t]{14.55cm} \color{red}
				\vspace{25pt}
				
				$$g\left(x, y, \frac{p}{q}\right) = x + 0y + 0p + 0q$$
				
				\vspace{25pt}
			\end{minipage}
		}
	
	
	
\clearpage
		
		\part[1] What element of the domain of $g$ maps to $23571113171923$ under the rule you specified in (\ref{rule-g})?
		
		\fbox{\begin{minipage}[t]{14.55cm} \color{red}
				\vspace{25pt}
				
				\begin{center}
					If we take any $x$ to be 23571113171923, and $y$, $p$, and $q$ to be any arbitrary elements $a$, $b$, and $c$ respectively within their respective domains, we have: $g\left(x, y, \frac{p}{q}\right) = g\left(23571113171923, a, \frac{b}{c}\right) 23571113171923 + 0a + 0b + 0c = 23571113171923$. As we can see, when $x = 23571113171923$, $y = a$, $p = b$, and $q = c$, $g$ maps to 23571113171923.
				\end{center}
				
				\vspace{25pt}
			\end{minipage}
		}
	
	
	
\clearpage
	
	
	
		\part[1] Prove that the rule that you specified in (\ref{rule-g}) is onto. If your rule is not onto, then you will automatically receive 0 for this part. \\
		
		\fbox{\begin{minipage}[t]{14.55cm} \color{red}
				\vspace{25pt}
				
				\begin{center}
				Here is our rule: $g\left(x, y, \frac{p}{q}\right) = x + 0y + 0p + 0q$. As we can see, this is the same as $g\left(x, y, \frac{p}{q}\right) = x$ by the multiplication property of zero. Since the values of $y$, $p$, and $q$ do not matter, the only numbers that this function can output are elements from the domain of $x$. Since the domain of $x$ is $\mathbb{N}$, and the codomain is also $\mathbb{N}$, the rule specified in (a) is onto, because each element of the codomain can be mapped to from (an) element(s) from the domain. $\square$
			\end{center}
				
				\vspace{25pt}
			\end{minipage}
		}
		
	\end{parts}
	
\clearpage

	
	\question[1] Finally, is there a bijection from $\mathbb{N} \times \mathbb{N} \times \mathbb{Q}^{\geq 0}$ into $\mathbb{N}$? Either way, justify your answer.
		
	\fbox{\begin{minipage}[t]{14.55cm} \color{red}
			\vspace{25pt}
			
			\begin{center}
				No, there is no bijection. I cannot even find a one-to-one function, even though finding an onto function is trivial. 4 variables makes it even more difficult to find unique outputs.
			\end{center}
			
			\vspace{25pt}
		\end{minipage}
	}
	
		
\end{questions}


\end{document}
